% !TeX root = main.tex
\documentclass[number,times]{elsarticle}

% \usepackage[utf8]{inputenc}
\usepackage[dvipsnames,svgnames,x11names,hyperref]{xcolor}

% Main packages
\usepackage[utf8]{inputenc}
\usepackage{amsmath}
\usepackage{amssymb}
\usepackage{physics} % For partial derivatives
\usepackage{bbm}
\usepackage[makeroom]{cancel}
\usepackage[textsize=tiny]{todonotes}
\usepackage[hyphens]{url}
\usepackage{hyperref}
\hypersetup{
 colorlinks=true,
 linkcolor=black,
 filecolor=black,
 urlcolor=black,
 citecolor=OliveGreen,
}

% Bibliography
% \usepackage[round,numbers]{natbib} % Not used since loaded by elsarticle

% For graphics
\usepackage{graphicx}

% For tikz figures
\usepackage{tikz}
\usetikzlibrary{positioning,calc} % I think for \matrix command and also for at=($()$) expression
\usetikzlibrary{shapes.geometric}
\usetikzlibrary{decorations.pathreplacing, calligraphy} % For curly braces
\usetikzlibrary{shapes, arrows.meta, positioning}
\usetikzlibrary{intersections} % Not sure if background needed?
\tikzstyle{arrow} = [thin, ->, >=stealth]
\tikzset{
    connect/.style args={(#1) to (#2) over (#3) by #4}{
        insert path={
            let \p1=($(#1)-(#3)$), \n1={veclen(\x1,\y1)}, 
            \n2={atan2(\y1,\x1)}, \n3={abs(#4)}, \n4={#4>0 ?180:-180}  in 
            (#1) -- ($(#1)!\n1-\n3!(#3)$) 
            arc (\n2:\n2+\n4:\n3) -- (#2)
        }
    },
}
\usetikzlibrary{shapes.geometric, arrows}
% \tikzstyle{startstop} = [rectangle, rounded corners, minimum width=3cm, minimum height=1cm,text centered, draw=black, fill=red!30]
\tikzstyle{io} = [trapezium, trapezium left angle=70, trapezium right angle=110, text width=0.3\columnwidth, minimum height=1cm, text centered, draw=black, fill=blue!30]
\tikzstyle{process} = [rectangle, minimum height=1cm, text centered, draw=black, fill=orange!30]
\tikzstyle{arrow} = [thick,->,>=stealth]
\tikzstyle{dashedarrow} = [thick,dashed,->,>=stealth]
\tikzstyle{box} = [draw, rectangle, minimum height=0.05\textwidth, minimum width=0.05\textwidth, align=left]
\tikzstyle{thinbox} = [draw, rectangle, minimum height=0.05\textwidth, minimum width=0.008\textwidth, align=left]
\newcommand{\dayplotheight}{0.05\textwidth}

% For nicer tables
\usepackage{booktabs}
\usepackage{csvsimple}
\usepackage{rotating} % For sideways tables
\usepackage{pdflscape} % To rotate pdf for one page 
\usepackage{longtable} % Since review table is too long otherwise
\usepackage{tabularx} % Automatically adjusts column widths (hopefully well../)

% For plotting
\usepackage{pgfplots}
\usepgfplotslibrary{groupplots,units} % For groupplots
\usepackage{pgfplotstable} % For plotting csv files as tables
\usepgfplotslibrary{statistics} % For boxplots
\usepgfplotslibrary{colorbrewer} % Nice color set
\pgfplotsset{
 compat=newest,
 % initialize Dark2
 cycle list/Dark2,
 % combine it with 'mark list*':
 cycle multiindex* list={
  mark list*\nextlist
  Dark2\nextlist
 },
} % Mainly for cycling through lists
\usepgfplotslibrary{groupplots} % For groupplots

% Style to select only points from #1 to #2 (inclusive)
% Useful for not having too many csv files
\pgfplotsset{select coords between index/.style 2 args={
    x filter/.code={
        \ifnum\coordindex<#1\def\pgfmathresult{}\fi
        \ifnum\coordindex>#2\def\pgfmathresult{}\fi
    }
}}


% For caching tikz figures
% \usetikzlibrary{external}
% \tikzexternalize[prefix=pgfplots_figs/]

\usepackage{float}

% For si units
\usepackage{siunitx}
% \NewDocumentCommand{\NaN}{}{\text{NaN}} % Attempt to also be able to parse nans
\sisetup{
 round-mode=figures,
 round-precision=3,
 % input-symbols=\NaN
} % Use \num{...} to round

% For creating a frame around the nomenclature
\usepackage{framed}

% For spaces after commands (see simulation names)
\usepackage{xspace}

% For subfigures
\usepackage[font={scriptsize}]{subcaption}

% To read csv files into e.g. tables
\usepackage{csvsimple}

% For the euro sign
\usepackage[gen]{eurosym} % use \EUR or \euro

% For \verb++ or \verbatim environments
\usepackage{verbatim}

% For cross and checkmarks
\usepackage{pifont}
\newcommand{\xmark}{\ding{55}}

% Acronyms
\usepackage{acronym}
\AtBeginDocument{%
  \renewcommand*{\AC@hyperlink}[2]{%
    \begingroup
      \hypersetup{hidelinks}%
      \hyperlink{#1}{#2}%
    \endgroup
  }
}

% Inserting code
\usepackage{listings}
\lstset{
basicstyle=\small\ttfamily,
columns=flexible,
breaklines=true
}

% Nomenclature
\usepackage{nomencl}
% \makenomenclature % Very important!

\usepackage{etoolbox} % For comparing prefixes
\renewcommand\nomgroup[1]{%
 \item[\bfseries
 \ifstrequal{#1}{A}{Sets}{%
 \ifstrequal{#1}{B}{Variables}{%
 \ifstrequal{#1}{C}{Parameters}{}}}%
]}

% You can specify parts of text which become e.g. two or three column
\usepackage{multicol}

% \newcommand{\DimensUnits}[1]{\hfill\makebox[2em]{#1\hfill}\ignorespaces}
% \newcommand{\nomdescr}[1]{\parbox[t]{10cm}{\RaggedRight #1}}
\newcommand{\nomwithunit}[4]{\nomenclature[#1]{#2}{#3 #4}}

% Font stuff and some defaults
% \usepackage[cm]{sfmath} % for sans serif maths
% \renewcommand{\rmdefault}{cmr}
% \renewcommand{\sfdefault}{cmss}
% \renewcommand{\ttdefault}{cmtt}
% \renewcommand{\familydefault}{\sfdefault}
% \usepackage[T1]{fontenc}
\setlength{\parskip}{\smallskipamount}
\setlength{\parindent}{0pt}
% \usepackage{babel}

\usepackage{array} % Define new column types!
\newcolumntype{L}[1]{>{\raggedright\let\newline\\\arraybackslash\hspace{0pt}}m{#1}}
\newcolumntype{C}[1]{>{\centering\let\newline\\\arraybackslash\hspace{0pt}}m{#1}}
\newcolumntype{R}[1]{>{\raggedleft\let\newline\\\arraybackslash\hspace{0pt}}m{#1}}
\renewcommand{\arraystretch}{1}

% Redefine vector to be bold font, not arrow on top
\let\vec\mathbf

% For drawing a grid
\makeatletter
\def\grd@save@target#1{%
  \def\grd@target{#1}}
\def\grd@save@start#1{%
  \def\grd@start{#1}}
\tikzset{
  grid with coordinates/.style={
    to path={%
      \pgfextra{%
        \edef\grd@@target{(\tikztotarget)}%
        \tikz@scan@one@point\grd@save@target\grd@@target\relax
        \edef\grd@@start{(\tikztostart)}%
        \tikz@scan@one@point\grd@save@start\grd@@start\relax
        \draw[minor help lines] (\tikztostart) grid (\tikztotarget);
        \draw[major help lines] (\tikztostart) grid (\tikztotarget);
        \grd@start
        \pgfmathsetmacro{\grd@xa}{\the\pgf@x/1cm}
        \pgfmathsetmacro{\grd@ya}{\the\pgf@y/1cm}
        \grd@target
        \pgfmathsetmacro{\grd@xb}{\the\pgf@x/1cm}
        \pgfmathsetmacro{\grd@yb}{\the\pgf@y/1cm}
        \pgfmathsetmacro{\grd@xc}{\grd@xa + \pgfkeysvalueof{/tikz/grid with coordinates/major step}}
        \pgfmathsetmacro{\grd@yc}{\grd@ya + \pgfkeysvalueof{/tikz/grid with coordinates/major step}}
        \foreach \x in {\grd@xa,\grd@xc,...,\grd@xb}
        \node[anchor=north] at (\x,\grd@ya) {\pgfmathprintnumber{\x}};
        \foreach \y in {\grd@ya,\grd@yc,...,\grd@yb}
        \node[anchor=east] at (\grd@xa,\y) {\pgfmathprintnumber{\y}};
      }
    }
  },
  minor help lines/.style={
    help lines,
    step=\pgfkeysvalueof{/tikz/grid with coordinates/minor step}
  },
  major help lines/.style={
    help lines,
    line width=\pgfkeysvalueof{/tikz/grid with coordinates/major line width},
    step=\pgfkeysvalueof{/tikz/grid with coordinates/major step}
  },
  grid with coordinates/.cd,
  minor step/.initial=.2,
  major step/.initial=1,
  major line width/.initial=2pt,
}

% Colors
\usepackage{color} % To define colours?

% Commands

% Other
\renewcommand{\eqref}[1]{Eq. \ref{#1}}
\newcommand{\figref}[1]{Figure \ref{#1}}
\newcommand{\bref}[1]{(\ref{#1})}
\newcommand{\tabref}[1]{Table \ref{#1}}
\newcommand{\appref}[1]{Appendix \ref{#1}}
\newcommand{\secref}[1]{Section \ref{#1}}

% Technologies
\newcommand{\CCGT}{CCGT\xspace}
\newcommand{\OCGT}{OCGT\xspace}
\newcommand{\WindOnsh}{Onshore Wind\xspace}
\newcommand{\WindOffsh}{Offshore Wind\xspace}
\renewcommand{\PV}{Solar\xspace}
\newcommand{\Battery}{Short Term\xspace}
\newcommand{\LTS}{Long Term\xspace}

% Storage operations
\newcommand{\mincost}{\emph{Min Cost}\xspace}
\newcommand{\ddvoll}{\emph{DDVOLL}\xspace}
\newcommand{\minlol}{\emph{Min LOL}\xspace}
\newcommand{\maxlol}{\emph{Max LOL}\xspace}
\newcommand{\minpeak}{\emph{Min Peak}\xspace}
\newcommand{\greedy}{\emph{Greedy}\xspace}

% Capacity credits
\newcommand{\avCC}{\emph{AvCC}\xspace}
\newcommand{\efceens}{\emph{EFC-EENS}\xspace}
\newcommand{\efclole}{\emph{EFC-LOLE}\xspace}

% Expectation operator
\newcommand{\Expct}{\mathop{\mathbb{E}}}

% Commands for comments
\usepackage{ulem} % For striking through text
\definecolor{SG}{rgb}{0,0,0} % Once it's done
\definecolor{citecol}{rgb}{0.5,0.5,0.5} % Once it's done
% \definecolor{SG}{rgb}{1,0,0} % Change later
\newcommand{\edst}[1]{} % For getting rid of strikethroughs
\newcommand{\ed}[1]{\textcolor{SG}{#1}}
% \textcolor{SG}{#1}
% \newcommand{\edst}[1]{\ed{\sout{#1}}}

\newcommand{\addcite}[1]{[\textcolor{citecol}{#1}]}

\newcommand{\paramessage}[1]{#1}
% \newcommand{\paramessage}[1]{1}

% \newcommand{\hide}[1]{\itshape #1 \normalfont}
\newcommand{\hide}[1]{}


\begin{document}

\title{Resource Adequacy and Operational Security Interaction in the EPOC 2030-50 Project}

\author{Sebastian Gonzato}

\maketitle

\newpage

\tableofcontents

\section*{Abbreviations}

\begin{acronym}
    \acro{UC}{Unit Commitment (Model)}
    \acro{DUCPR}{Deterministic Unit Commitment Model with Probabilistic Reserve constraints}
\end{acronym}


\newpage

\section{Interaction description}

Adequacy assessments and operational security analyses are two fundamental exercises in determining and ensuring the reliable operation of the electric power system. The former assesses the ability of the power system to satisfy demand for electricity in the day ahead stage, while the latter assesses (or ensures, depending on the context) that an unexpected event, such as an outage of a transmission line or generator or lower or greater than expected realisation of demand or renewable generation, does not lead to cascading failures and a possible blackout.

Typically, these exercises are performed independently. This interaction combines the two by feeding the results of an adequacy assessment to an operational security analysis.

In doing so, we deviate from typical practices in a number of ways. First, the adequacy assessment is done for a limited number of days instead of many Monte Carlo years for the sake of clarity and brevity. Indeed, doing this exercise for many Monte Carlo years would drastically increase the computational complexity while the aim of this exercise is to illustrate how such an interaction could be performed.

Secondly, the adequacy assessment is done using an unconventional \ac{UC} model, a modified version of the \ac{DUCPR} model described in \cite{Bruninx2017}. This model can be seen as a compromise between a computationally expensive stochastic \ac{UC} model and a more tractable though less accurate deterministic \ac{UC} model with operating reserve requirements. The treatment of operating reserves in this model allows a trade-off between shedding load in the day ahead stage, which would reduce day ahead adequacy, and shedding operating reserves, which should reduce operational security. By making this trade-off explicit and verifying the results in an operational security analysis we aim to elaborate on the results of \cite{Hermans2018}, which did not consider the possibility that shedding operating reserves would lead to insecure operation of the electric power system.

All code can be found at \href{https://github.com/junglegobs/ASoSEPOC}{https://github.com/junglegobs/ASoSEPOC}.

The rest of this document is organised as follows. \secref{sec:model} describes the \ac{DUCPR} model used to simulate power system operations and obtain day ahead adequacy and very approximate real time operational security indicators, namely scheduled load shedding and reserve shedding. \secref{sec:data} describes the stylised Belgian system analysed. \secref{sec:initial_results} analyses a single day in order to identify sources of adequacy and security issues, while \secref{sec:adequacy_vs_security} analyses the trade-off between these two.

\section{Description of a Deterministic Unit Commitment Model with Probabilistic Reserve (DUC-PR) constraints} \label{sec:model}

Sets, variables and costs related to unit commitment and storage operation have been omitted for brevity though they are included in the model. The interested reader is referred to \href{https://github.com/junglegobs/ASoSEPOC}{https://github.com/junglegobs/ASoSEPOC} for the relevant code.

\subsection{Sets}

\begin{itemize}
    \item $G$ - Generators
    \item $G_n$ - Generators located at node $n$
    \item $GD$ - Dispatchable / thermal / conventional generators
    \item $GR$ - Renewable / variable generators (\ac{VRES}).
    \item $N$ - Nodes or buses in the network
    \item $B$ - Lines or branches
    \item $T$ - Time steps / intervals / slices
    \item $L^+$ - Upward reserve levels
    \item $L^-$ - Downward reserve levels
\end{itemize}

\subsection{Parameters}

\begin{itemize}
    \item $D_{nt}$ - Demand
    \item $PTDF_{nl}$ - Power Transfer Distribution Factor
    \item $AF_{gt}$ - Availability Factor
    \item $K_g$ - Capacity
    \item $P^{min}_g$ - Minimum power output
    \item $P^{max}_g$ - Maximum power output
    \item $D^{L^+}_{lt}$ - Upward reserve requirement
    \item $D^{L^-}_{lt}$ - Downward reserve requirement
\end{itemize}

\subsection{Variables}

All variables are positive apart from node injection variables which may also be negative.

\begin{itemize}
    \item $q_{gt}$ - Generation
    \item $\hat{q}_{gt}$ - Generation above the minimum stable operating point (0 in the case of renewables).
    \item $ls_{nt}$ - Load shedding
    \item $inj_{nt}$ - Node injection
    \item $f_{lt}$ - Branch flow
    \item $r^+_{gt}$ - Total upward reserve provision
    \item $r^-_{gt}$ - Total downward reserve provision
    \item $r^{L^+}_{glt}$ - Upward reserve provision for reserve level $l$
    \item $r^{L^-}_{glt}$ - Downward reserve provision for reserve level $l$
    \item $rs_{nlt}$ - Upward reserve shedding for reserve level $l$
    \item $rc_{nlt}$ - Downward reserve provided by day ahead load shedding for reserve level $l$
    \item $rinj_{nlt}^{L^+}$ - Possible node injection due to activation of upward reserve level $l$
    \item $rinj_{nlt}^{L^-}$ - Possible node injection due to activation of downward reserve level $l$
    \item $rf_{nbt}^{L^+}$ - Possible branch flow due to activation of upward reserve level $l$
    \item $rf_{nbt}^{L^-}$ - Possible branch flow due to activation of downward reserve level $l$
    \item $d_{nlt}^{L^+}$ - Possible imbalance on node $n$ for upward reserve level $l$
    \item $d_{nlt}^{L^-}$ - Possible imbalance on node $n$ for downward reserve level $l$
\end{itemize}

\subsection{Objective}

\begin{align}
    \min \quad & \sum_{t \in T} \sum_{g \in G} C^{var}_{g} \cdot \hat{q}_{gt} \nonumber                                                                  \\
               & + \sum_{t \in T} \sum_{l \in L^+} \sum_{n \in N} P^{L^+} \cdot \sum_{g \in G} C^{var}_{g} \cdot r^{L^+}_{glt} + C^{shed} \cdot rs_{nlt} \\
               & - \sum_{t \in T} \sum_{l \in L^-} \sum_{n \in N} P^{L^-} \cdot C^{var}_{g} \cdot r^{L^-}_{gnlt} + C^{shed} \cdot rc_{nlt} \nonumber
\end{align}

From the top line to the bottom, the costs are those of dispatching generators and activating upwards or downwards reserves.

\subsection{Constraints}

\subsubsection{Day ahead constraints}

The power balance:

\begin{equation}
    \sum_{g \in G_n} q_{gt} + ls_{nt} = D_{nt} + inj_{nt} \quad n \in N, \; t \in T
\end{equation}

Note the use of the set $G_n$ to only allow generators at node $n$ to contribute to the power balance. Another way of describing this would have been through an incidence matrix.

Network constraints:

\begin{align}
    f_{bt} = \sum_{n \in N} PTDF_{nb} \cdot inj_{nt} & \quad b \in B, \; t \in T \\
    -F_b \leq f_{bt} \leq F_b                        & \quad b \in B, \; t \in T \\
    \sum_{n \in N} inj_{nt} = 0                      & \quad n \in N, \; t \in T \\
\end{align}

Constraints on generator output:

\begin{align}
    q_{gt} - r^{-}_{gt} \geq 0                    & \quad g \in GR, \; t \in T \\
    q_{gt} + r^{+}_{gt} \leq AF_{gt} \cdot K_g    & \quad g \in GR, \; t \in T \\
    q_{gt} - r^{-}_{gt} \geq P^{min} \cdot z_{gt} & \quad g \in GD, \; t \in T \\
    q_{gt} + r^{+}_{gt} \leq P^{max} \cdot z_{gt} & \quad g \in GD, \; t \in T
\end{align}

For brevity and clarity, constraints on ramping and minimum up and down times are omitted.

\subsubsection{Operating reserve (activation) constraints}

The constraints on reserve provision are as follows:

\begin{align}
    D^{L^+}_{lt} = \sum_{g \in G} r^{L^+}_{glt} + \sum_{n \in N} rs_{nlt} & \quad l \in L^+, \; t \in T \\
    D^{L^-}_{lt} = \sum_{g \in G} r^{L^-}_{glt} + \sum_{n \in N} rc_{nlt} & \quad l \in L^-, \; t \in T \\
    \sum_{l \in L^-} rc_{nlt} \leq ls_{nt}                                & \quad n \in N, \; t \in T   \\
    r^{+}_{gt} = \sum_{l \in L^+} r^{L^+}_{gnlt}                          & \quad g \in G, \; t \in T   \\
    r^{-}_{gt} = \sum_{l \in L^-} r^{L^-}_{gnlt}                          & \quad g \in G, \; t \in T
\end{align}

The amount of reserve shedding can be limited to a fraction of the total upward reserve requirements $RSL$:

\begin{equation}
    \sum_{n \in N, l \in L^+} rs_{nlt} \leq RSL \cdot \sum_{l \in L^+} D^{L^+}_{l't} \quad t \in T
\end{equation}

There are several matters to note here:

\begin{itemize}
    \item The operating reserve balance is performed over the entire network, not per node.
    \item The operating reserve balance is split into reserve levels. Higher reserve levels (values of $l$) are less likely to occur.
    \item It is possible to shed upward reserves, and this is more likely to occur for higher reserve levels. This model is therefore able to make a trade-off between day ahead adequacy and real time operational security, albeit crudely.
    \item Shedding load in day ahead allows additional downward reserves to be provided through the variable $rc_{nlt}$. Implicitly this assumes that load can be `activated' in real time to provide downward reserves.
\end{itemize}

\subsubsection{Operating reserve activation network constraints} \label{sec:operating_reserve_network_activation_constraints}

The following constraints attempt to take network constraints into account (albeit very weakly):

\begin{align}
    \sum_{g \in G_n, l'=1:l} (r^{L^+}_{glt} + rs_{nlt}) = d^{L^+}_{nlt} + rinj^{L^+}_{nlt}   & \quad n \in N, \; l \in L^+, \; t \in T \label{eq:reserve_network_activation_constraints_1} \\
    - \sum_{g \in G_n, l'=1:l} (r^{L^-}_{glt} + rc_{nlt}) = d^{L^-}_{nlt} + rinj^{L^-}_{nlt} & \quad n \in N, \; l \in L^-, \; t \in T                                                     \\
    \sum_{n \in N} d^{L^+}_{nlt} = \sum_{l'=1:l} D^{L^+}_{l't}                               & \quad l \in L^+, \; t \in T                                                                 \\
    \sum_{n \in N} d^{L^-}_{nlt} = - \sum_{l'=1:l} D^{L^-}_{l't}                             & \quad l \in L^-, \; t \in T                                                                 \\
    rf^{L^+}_{blt} = \sum_{n \in N} PTDF_{nb} \cdot rinj^{L^+}_{nlt}                         & \quad b \in B, \; l \in L^+, \; t \in T                                                     \\
    rf^{L^-}_{blt} = \sum_{n \in N} PTDF_{nb} \cdot rinj^{L^-}_{nlt}                         & \quad b \in B, \; l \in L^-, \; t \in T                                                     \\
    -F_{b} \leq f_{bt} + rf^{L^+}_{blt} \leq F_b                                             & \quad b \in B, \; l \in L^+, \; t \in T                                                     \\
    -F_{b} \leq f_{bt} + rf^{L^-}_{blt} \leq F_b                                             & \quad b \in B, \; l \in L^-, \; t \in T                                                     \\
    \sum_{n \in N} rinj^{L^+}_{nt} = 0                                                       & \quad l \in L^+, \; n \in N, \; t \in T                                                     \\
    \sum_{n \in N} rinj^{L^-}_{nt} = 0                                                       & \quad l \in L^-, \; n \in N, \; t \in T \label{eq:reserve_network_activation_constraints_2}
\end{align}

Note that $rinj^{L^+}_{nlt}$, $rinj^{L^-}_{nlt}$, $d^{L^+}_{nlt}$ and $d^{L^-}_{nlt}$ are all free variables and there is a change in sign between upward and downward reserve levels to ensure that the resulting additional line flows are correct.

Since imbalances are aggregated across the network, a particular reserve level activation is not associated with an imbalance at the nodal level. The above constraints therefore enforce that for each reserve level $l$ and node $n$, there exists some combination of nodal imbalance, node injections, generator dispatches and line flows which would satisfy the network constraints AND the imbalance across the entire network.

\subsubsection{Further constraining reserve activation network constraints} \label{sec:further_constraining_operating_reserve_network_activation_constraints}

Given the formulation here, which uses reserve levels, i.e. quantiles, over the entire network to represent forecast errors, it is difficult to come up with more stringent conditions. However, we devised two ways of doing so. The first is to apply box constraints to the possible nodal imbalances:

\begin{align}
    d_{nt}^{min} \leq d_{nlt}^{L^+} \leq d_{nt}^{max} & \quad n \in N, \; l \in L^+, \; t \in T \\
    d_{nt}^{min} \leq d_{nlt}^{L^-} \leq d_{nt}^{max} & \quad n \in N, \; l \in L^-, \; t \in T
\end{align}

It is possible to calculate $d_{nt}^{min}$ and $d_{nt}^{max}$ since the quantiles for $D^{L^+}$ and $D^{L^-}$ are calculated based on forecast error scenarios which \emph{are} defined at the nodal level. These limits are shown in Figure \ref{fig:imbalance_range}. Clearly it is simply not possible to have an imbalance on some nodes, such as Coo (the location of Belgium's pumped hydro unit). Constraining $d_{nlt}^{L^+}$ and $d_{nlt}^{L^-}$ in this way is referred to as AbsImb later on.

\begin{figure}[ht]
    \centering
    % \includegraphics[width=0.9\textwidth]{../plots/nodal_imbalance_investigation/imbalance_range}
    \caption{Range of possible imbalances for all hours of day 309.\label{fig:imbalance_range}}
\end{figure}

The other possibility is to restrict $d_{nlt}^{L^+}$ and $d_{nlt}^{L^-}$ to lie within the convex hull of all imbalances. This is illustrated in 2 dimensions in Figure \ref{fig:convex_hull_tihange} for the imbalances at Tihange 1 and 2. Clearly the imbalances are highly correlated, and constraining the possible nodal imbalances to lie within the convex hull exploits this in a way that box constraints would not be able to.

\begin{figure}[ht]
    \centering
    % \includegraphics[width=0.9\textwidth]{../plots/nodal_imbalance_investigation/convex_hull_TIHANGE 1_TIHANGE 2}
    \caption{Illustration of convex hull constraints on $d_{nlt}^{L^+}$ and $d_{nlt}^{L^-}$. Blue circles are imbalance scenarios, shaded area is the convex hull, $d_{nt}^{min}$ and $d_{nt}^{max}$ are the x limits for TIHANGE 1 and the y limits for TIHANGE 2. \label{fig:convex_hull_tihange}}
\end{figure}

At the time of writing, this convex hull restriction led to model infeasibilities and so it is not

\section{Input data - stylised Belgian grid with a large amount of variable renewable energy sources} \label{sec:data}

All data used for this exercise can be found at: \href{https://www.dropbox.com/sh/mdvmc082gwng0tr/AABRyc3fZpxAFycmUfZmh8Csa?dl=0}.

The grid and resource data used for this interaction is inspired by the case study in \cite{Belderbos2020}, omitting the gas network and power to gas technologies. This data gives a stylised Belgian system with a high penetration (~8s0\%) of \ac{VRES}, which includes 75.7 GW of solar PV, 7.0 GW of Onshore Wind and 7.3 GW of Offshore Wind. The grid consists of 46 nodes connected by 69 lines. The total amount of conventional thermal generation is 9.9 GW, with all generators based on new \ac{CCGT} units apart from one new \ac{OCGT} unit at Lixhe. The nuclear generators of Doel and Tihange are therefore replaced with new \ac{CCGT} units.

Power to Gas is omitted but 14.1 GW of storage power capacity (1.1 GW provided by pumped hydro at Coo, the rest batteries) and 101 GWh of storage energy capacity (8.2 GWh provided by pumped hydro at Coo, the rest batteries) is included. The batteries therefore have a duration of 7.2 hours (which could be considered atypical, given current durations of 1 - 2 hours \addcite{MIT paper for this?}).

The residual load time series, aggregated across all nodes, for this system is plotted in \todo[inline]{Plot this}.

A complete overview of the data is given in \cite{Belderbos2020}, Appendix B.

\section{Analysis of \acl{DUCPR} model results} \label{sec:initial_results}

This analysis relates only to the 309th day of the year (November 5th). The aim is to investigate what are the causes of load shedding.

\subsection{No operating reserves}

\tabref{tab:results_no_OR} lists results for an increasing level of technical constraints or increasingly inflexible system. While the objective more than doubles going from the simple linear economic dispatch model with no network constraints to the unit commitment model with network constraints, no load shedding occurs. Preventing simultaneous charging and discharging constraints does not affect results at all, implying that additional energy consumption (which is not physically possible) from storage does not aid congestion in this context.

\begin{table}[H]
    \centering
    \begin{tabular}{ccccc}
        \toprule
        UC  & DANet & PSCD & Load shedding [MWh] & Objective \\
        \midrule
            &       &      & 0.0                 & 101,915   \\
        \xm &       &      & 0.0                 & 109,148   \\
        \xm & \xm   &      & 0.0                 & 227,495   \\
        \xm & \xm   & \xm  & 0.0                 & 227,492   \\
        \bottomrule
    \end{tabular}
    \caption{Analysis of model results when operating reserves are not included. UC = Unit Commitment constraints, DANet = Day Ahead Network constraints, PSCD = Prevent Simultaneous Charging and Discharging constraints.}\label{tab:results_no_OR}
\end{table}

The dispatch for the model with and without unit commitment constraints and with network constraints is shown in \figref{fig:dispatch_base}. The addition of unit commitment constraints leads to storage dispatching and conventional units generating throughout the day. Interestingly, the amount of solar generation and storage dispatch is higher in the economic dispatch model and the model with unit commitment and network constraints. This may be because batteries and solar generators are located in buses where there is also load, favouring their usage.

\begin{figure}[H]
    \centering
    \begin{subfigure}[t]{0.7\textwidth}
        \centering
        % \includegraphics[width=\textwidth]{../data/sims/main_model_runs/309/base/dispatch}
        \caption{Economic dispatch model}
    \end{subfigure}
    \begin{subfigure}[t]{0.7\textwidth}
        \centering
        % \includegraphics[width=\textwidth]{../data/sims/main_model_runs/309/base_UC=true/dispatch}
        \caption{Unit commitment model}
    \end{subfigure}
    \begin{subfigure}[t]{0.7\textwidth}
        \centering
        % \includegraphics[width=\textwidth]{../data/sims/main_model_runs/309/base_UC=true_DANet=true/dispatch}
        \caption{Unit commitment model with day ahead network constraints}
    \end{subfigure}
    \caption{Dispatch schedules with and without operating reserves.\label{fig:dispatch_base}}
\end{figure}

\subsection{Probabilistic operating reserves without reserve activation network constraints}

\begin{itemize}
    \item Unit commitment constraints lead to unavoidable load shedding.
    \item This result should be interpreted with caution however, since reserve provision is quite strict (storage can't provide reserves, non spinning reserves not modelled). If more flexibility would be available in terms of reserve provision then this might not occur.
    \item Network constraints are less problematic than unit commitment (cost increase of \~20,000 euros instead of \~500,000).
    \item Preventing simultaneous charge and discharge has a small but noticeable effect.
\end{itemize}

\begin{table}[ht]
    \centering
    \footnotesize
    \begin{tabular}{ccccccc}
        \toprule
        UC  & DANet & PSCD & RSL & Load shedding [MWh] & Reserve Shedding [MWh] & Objective \\
        \midrule
            &       &      & 0.0 & 0.0                 & 0.0                    & 101,345   \\
            &       &      & 0.5 & 0.0                 & 0.0                    & 101,014   \\
            &       &      & 1.0 & 0.0                 & 0.0                    & 101,014   \\
        \midrule
        \xm &       &      & 0.0 & 1.669               & 0.0                    & 617,536   \\
        \xm &       &      & 0.5 & 1.669               & 0.0                    & 617,536   \\
        \xm &       &      & 1.0 & 1.669               & 0.0                    & 617,536   \\
        \midrule
        \xm & \xm   &      & 0.0 & 1.669               & 0.0                    & 650,647   \\
        \xm & \xm   &      & 0.5 & 1.669               & 136                    & 650,231   \\
        \xm & \xm   &      & 1.0 & 1.669               & 136                    & 650,231   \\
        \midrule
        \xm & \xm   & \xm  & 0.0 & 1.669               & 0.0                    & 650,647   \\
        \xm & \xm   & \xm  & 0.5 & 1.669               & 136                    & 650,231   \\
        \xm & \xm   & \xm  & 1.0 & 1.669               & 136                    & 650,263   \\
        \bottomrule
    \end{tabular}
    \caption{Analysis of model results with operating reserves but no reserve activation network constraints. UC = Unit Commitment constraints, DANet = Day Ahead Network constraints, PSCD = Prevent Simultaneous Charging and Discharging constraints, RSL = Reserve Shedding Limit.}\label{tab:results_no_RANet}
\end{table}

\figref{fig:dispatch_simple_reserves} shows the dispatch for 3 models with operating reserves, the first a simple economic dispatch, then a unit commitment and then a unit commitment with network constraints. Similarly to before, network constraints lead to greater usage of solar and storage compared to the unit commitment model without.

\begin{figure}[H]
    \centering
    \begin{subfigure}[t]{0.7\textwidth}
        \centering
        % \includegraphics[width=\textwidth]{../data/sims/main_model_runs/309/base_RSV=0.0/dispatch}
        \caption{No UC, no DANet, no PSCD, RSL is 0.0}
    \end{subfigure}
    \begin{subfigure}[t]{0.7\textwidth}
        \centering
        % \includegraphics[width=\textwidth]{../data/sims/main_model_runs/309/base_UC=true_RSV=0.0/dispatch}
        \caption{UC, no DANet, no PSCD, RSL is 0.0}
    \end{subfigure}
    \begin{subfigure}[t]{0.7\textwidth}
        \centering
        % \includegraphics[width=\textwidth]{../data/sims/main_model_runs/309/base_UC=true_DANet=true_RSV=0.0/dispatch}
        \caption{UC, DANet, no PSCD, RSL is 0.0}
    \end{subfigure}
    \caption{Dispatch schedules with and without operating reserves.\label{fig:dispatch_simple_reserves}}
\end{figure}

\begin{itemize}
    \item What's happening to reduce reserve shedding without reducing load shedding? Consider only the difference between $RSL$ 0.5 and 0.0 for the UC and DANet case.
    \item For $RSL = 0.5 $, load shedding occurs across all nodes for the 10th hour of the day. Reserve shedding occurs on one node, LATOUR, for the 9th hour.
    \item The commitment for the two values of $RSL$ is not different.
    \item The load shed in both cases is the same i.e. it is not shed in different places. It is also equal everywhere, which seems strange...
    \item To avoid reserve shedding, the cost increase appears to be simply from regulating the generation of 6 conventional generators - Eeklo Nord, Meerhout, Monce and the generators at TIHANGE 1 and TIHANGE 3. Various renewable generators also modify their output.
    \item This leads me to believe that economic load shedding is occurring, and that more conventional generators are generating and providing reserves in order to satisfy the reserve shedding limit.
\end{itemize}

The fact that upward operating reserves can be entirely satisfied while load shedding occurs signals to me that reserve provision is easier to provide than satisfying load. This is almost certainly due to the weak reserve activation network constraints. Further constraining these using the methods described in \secref{sec:further_constraining_operating_reserve_network_activation_constraints} attempts to address this, as will be detailed in the next section.

\subsection{Probabilistic operating reserves with reserve activation network constraints}

Including network reserve activation constraints, that is Constraints \ref{eq:reserve_network_activation_constraints_1} through \ref{eq:reserve_network_activation_constraints_1}, led to no discernible change in objective. This particular instance is not able to make a trade-off between day ahead adequacy and real time operational security. As a first attempt to overcome this issue, the load at all nodes was multiplied by a factor of 1.5. This gave the results shown in \tabref{tab:results_LM_1_5}.

\begin{table}[ht]
    \centering
    \footnotesize
    \begin{tabular}{cccccccc}
        \toprule
        UC  & DANet & PSCD & AbsImb & RSL & Load shedding [MWh] & Reserve Shedding [MWh] & Objective \\
        \midrule
            &       &      &        & 0.0 & 32,366              & 0                      & 7,290,909 \\
            &       &      &        & 0.5 & 0.0                 & 36,269                 & 447,102   \\
            &       &      &        & 1.0 & 0.0                 & 36,082                 & 440,467   \\
        \midrule
        \xm &       &      &        & 0.0 & 32,366              & 0                      & 7,433,785 \\
        \xm &       &      &        & 0.5 & 1,669               & 34,379                 & 924,686   \\
        \xm &       &      &        & 1.0 & 1,669               & 34,308                 & 919,363   \\
        \midrule
        \xm & \xm   &      &        & 0.0 & 33,929              & 0                      & 7,822,959 \\
        \xm & \xm   &      &        & 0.5 & 1,967               & 39,234                 & 1,249,422 \\
        \xm & \xm   &      &        & 1.0 & 1,941               & 39,160                 & 1,245,647 \\
        \midrule
        \xm & \xm   & \xm  &        & 0.0 & 33,929              & 0                      & 7,822,959 \\
        \xm & \xm   & \xm  &        & 0.5 & 1,967               & 39,234                 & 1,249,422 \\
        \xm & \xm   & \xm  &        & 1.0 & 1,941               & 39,160                 & 1,245,647 \\
        \midrule
        \xm & \xm   & \xm  & \xm    & 0.0 & 33,929              & 0                      & 7,822,959 \\
        \xm & \xm   & \xm  & \xm    & 0.5 & 1,967               & 39,234                 & 1,249,422 \\
        \xm & \xm   & \xm  & \xm    & 1.0 & 1,941               & 39,160                 & 1,245,647 \\
        \bottomrule
    \end{tabular}
    \caption{Analysis of model results with operating reserves and with reserve activation network constraints with load multiplied by a factor of 1.5. UC = Unit Commitment constraints, DANet = Day Ahead Network constraints, PSCD = Prevent Simultaneous Charging and Discharging constraints, RSL = Reserve Shedding Limit.}\label{tab:results_no_RANet}
\end{table}

\todo[inline]{Objective is weird for some cases...}

With reference to \tabref{tab:results_LM_1_5}:

\begin{itemize}
    \item Even in this case, the reserve activation network constraints are not binding, since the AbsImb constraints do not alter the objective function.
    \item On the other hand, the trade-off between reserve shedding and load shedding can be seen now in all cases. This trade-off is also obvious when observing the objective function.
    \item Unit commitment constraints do relatively little to increase total costs, but this is unsurprising given that costs are dominated by load shedding.
    \item Day ahead network constraints increase load shedding considerably i.e. congestion also plays a role in increasing load shedding.
\end{itemize}

Closer investigation at load shedding patterns for the model with UC and DANet (but no PSCD and AbsImb) and $RSL = 0.0$ reveals that load shedding is at it's highest in the mornings (8 - 10 AM) and in the evenings (17 - 20). No load shedding occurs in the middle of the day, which happens to be when load is lowest and solar generation is greatest. Load shedding is roughly equally distributed throughout the day in absolute though not relative terms.

\begin{figure}[H]
    \centering
    \begin{subfigure}[t]{0.45\textwidth}
        \centering
        % \includegraphics[width=\textwidth]{../plots/main_model_runs/ls_timeseries_idx=56}
        \caption{}
    \end{subfigure}
    \newline
    \begin{subfigure}[t]{0.45\textwidth}
        \centering
        % \includegraphics[width=\textwidth]{../plots/main_model_runs/ls_per_node_abs_idx=56}
        \caption{}
    \end{subfigure}
    \begin{subfigure}[t]{0.45\textwidth}
        \centering
        % \includegraphics[width=\textwidth]{../plots/main_model_runs/ls_per_node_idx=56}
        \caption{}
    \end{subfigure}
    \caption{Load shedding distribution in time (top figure, aggregated over all nodes) and space (aggregated over the entire day) for UC, DANet, LM = 1.5 and $RSL = 0.5$.}
\end{figure}

\section{Day ahead adequacy and real time operational security trade offs} \label{sec:adequacy_vs_security}

\end{document}
