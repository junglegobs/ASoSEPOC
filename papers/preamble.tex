% \usepackage[utf8]{inputenc}
\usepackage[dvipsnames,svgnames,x11names,hyperref]{xcolor}

% Main packages
\usepackage[utf8]{inputenc}
\usepackage{amsmath}
\usepackage{amssymb}
\usepackage{physics} % For partial derivatives
\usepackage{bbm}
\usepackage[makeroom]{cancel}
\usepackage[textsize=tiny]{todonotes}
\usepackage[hyphens]{url}
\usepackage{hyperref}
\hypersetup{
 colorlinks=true,
 linkcolor=black,
 filecolor=black,
 urlcolor=black,
 citecolor=OliveGreen,
}

% Bibliography
% \usepackage[round,numbers]{natbib} % Not used since loaded by elsarticle

% For graphics
\usepackage{graphicx}

% For tikz figures
\usepackage{tikz}
\usetikzlibrary{positioning,calc} % I think for \matrix command and also for at=($()$) expression
\usetikzlibrary{shapes.geometric}
\usetikzlibrary{decorations.pathreplacing, calligraphy} % For curly braces
\usetikzlibrary{shapes, arrows.meta, positioning}
\usetikzlibrary{intersections} % Not sure if background needed?
\tikzstyle{arrow} = [thin, ->, >=stealth]
\tikzset{
    connect/.style args={(#1) to (#2) over (#3) by #4}{
        insert path={
            let \p1=($(#1)-(#3)$), \n1={veclen(\x1,\y1)}, 
            \n2={atan2(\y1,\x1)}, \n3={abs(#4)}, \n4={#4>0 ?180:-180}  in 
            (#1) -- ($(#1)!\n1-\n3!(#3)$) 
            arc (\n2:\n2+\n4:\n3) -- (#2)
        }
    },
}
\usetikzlibrary{shapes.geometric, arrows}
% \tikzstyle{startstop} = [rectangle, rounded corners, minimum width=3cm, minimum height=1cm,text centered, draw=black, fill=red!30]
\tikzstyle{io} = [trapezium, trapezium left angle=70, trapezium right angle=110, text width=0.3\columnwidth, minimum height=1cm, text centered, draw=black, fill=blue!30]
\tikzstyle{process} = [rectangle, minimum height=1cm, text centered, draw=black, fill=orange!30]
\tikzstyle{arrow} = [thick,->,>=stealth]
\tikzstyle{dashedarrow} = [thick,dashed,->,>=stealth]
\tikzstyle{box} = [draw, rectangle, minimum height=0.05\textwidth, minimum width=0.05\textwidth, align=left]
\tikzstyle{thinbox} = [draw, rectangle, minimum height=0.05\textwidth, minimum width=0.008\textwidth, align=left]
\newcommand{\dayplotheight}{0.05\textwidth}

% For nicer tables
\usepackage{booktabs}
\usepackage{csvsimple}
\usepackage{rotating} % For sideways tables
\usepackage{pdflscape} % To rotate pdf for one page 
\usepackage{longtable} % Since review table is too long otherwise
\usepackage{tabularx} % Automatically adjusts column widths (hopefully well../)

% For plotting
\usepackage{pgfplots}
\usepgfplotslibrary{groupplots,units} % For groupplots
\usepackage{pgfplotstable} % For plotting csv files as tables
\usepgfplotslibrary{statistics} % For boxplots
\usepgfplotslibrary{colorbrewer} % Nice color set
\pgfplotsset{
 compat=newest,
 % initialize Dark2
 cycle list/Dark2,
 % combine it with 'mark list*':
 cycle multiindex* list={
  mark list*\nextlist
  Dark2\nextlist
 },
} % Mainly for cycling through lists
\usepgfplotslibrary{groupplots} % For groupplots

% Style to select only points from #1 to #2 (inclusive)
% Useful for not having too many csv files
\pgfplotsset{select coords between index/.style 2 args={
    x filter/.code={
        \ifnum\coordindex<#1\def\pgfmathresult{}\fi
        \ifnum\coordindex>#2\def\pgfmathresult{}\fi
    }
}}


% For caching tikz figures
% \usetikzlibrary{external}
% \tikzexternalize[prefix=pgfplots_figs/]

\usepackage{float}

% For si units
\usepackage{siunitx}
% \NewDocumentCommand{\NaN}{}{\text{NaN}} % Attempt to also be able to parse nans
\sisetup{
 round-mode=figures,
 round-precision=3,
 % input-symbols=\NaN
} % Use \num{...} to round

% For creating a frame around the nomenclature
\usepackage{framed}

% For spaces after commands (see simulation names)
\usepackage{xspace}

% For subfigures
\usepackage[font={scriptsize}]{subcaption}

% To read csv files into e.g. tables
\usepackage{csvsimple}

% For the euro sign
\usepackage[gen]{eurosym} % use \EUR or \euro

% For \verb++ or \verbatim environments
\usepackage{verbatim}

% For cross and checkmarks
\usepackage{pifont}
\newcommand{\xmark}{\ding{55}}

% Acronyms
\usepackage{acronym}
\AtBeginDocument{%
  \renewcommand*{\AC@hyperlink}[2]{%
    \begingroup
      \hypersetup{hidelinks}%
      \hyperlink{#1}{#2}%
    \endgroup
  }
}

% Inserting code
\usepackage{listings}
\lstset{
basicstyle=\small\ttfamily,
columns=flexible,
breaklines=true
}

% Nomenclature
\usepackage{nomencl}
% \makenomenclature % Very important!

\usepackage{etoolbox} % For comparing prefixes
\renewcommand\nomgroup[1]{%
 \item[\bfseries
 \ifstrequal{#1}{A}{Sets}{%
 \ifstrequal{#1}{B}{Variables}{%
 \ifstrequal{#1}{C}{Parameters}{}}}%
]}

% You can specify parts of text which become e.g. two or three column
\usepackage{multicol}

% \newcommand{\DimensUnits}[1]{\hfill\makebox[2em]{#1\hfill}\ignorespaces}
% \newcommand{\nomdescr}[1]{\parbox[t]{10cm}{\RaggedRight #1}}
\newcommand{\nomwithunit}[4]{\nomenclature[#1]{#2}{#3 #4}}

% Font stuff and some defaults
% \usepackage[cm]{sfmath} % for sans serif maths
% \renewcommand{\rmdefault}{cmr}
% \renewcommand{\sfdefault}{cmss}
% \renewcommand{\ttdefault}{cmtt}
% \renewcommand{\familydefault}{\sfdefault}
% \usepackage[T1]{fontenc}
\setlength{\parskip}{\smallskipamount}
\setlength{\parindent}{0pt}
% \usepackage{babel}

\usepackage{array} % Define new column types!
\newcolumntype{L}[1]{>{\raggedright\let\newline\\\arraybackslash\hspace{0pt}}m{#1}}
\newcolumntype{C}[1]{>{\centering\let\newline\\\arraybackslash\hspace{0pt}}m{#1}}
\newcolumntype{R}[1]{>{\raggedleft\let\newline\\\arraybackslash\hspace{0pt}}m{#1}}
\renewcommand{\arraystretch}{1}

% Redefine vector to be bold font, not arrow on top
\let\vec\mathbf

% For drawing a grid
\makeatletter
\def\grd@save@target#1{%
  \def\grd@target{#1}}
\def\grd@save@start#1{%
  \def\grd@start{#1}}
\tikzset{
  grid with coordinates/.style={
    to path={%
      \pgfextra{%
        \edef\grd@@target{(\tikztotarget)}%
        \tikz@scan@one@point\grd@save@target\grd@@target\relax
        \edef\grd@@start{(\tikztostart)}%
        \tikz@scan@one@point\grd@save@start\grd@@start\relax
        \draw[minor help lines] (\tikztostart) grid (\tikztotarget);
        \draw[major help lines] (\tikztostart) grid (\tikztotarget);
        \grd@start
        \pgfmathsetmacro{\grd@xa}{\the\pgf@x/1cm}
        \pgfmathsetmacro{\grd@ya}{\the\pgf@y/1cm}
        \grd@target
        \pgfmathsetmacro{\grd@xb}{\the\pgf@x/1cm}
        \pgfmathsetmacro{\grd@yb}{\the\pgf@y/1cm}
        \pgfmathsetmacro{\grd@xc}{\grd@xa + \pgfkeysvalueof{/tikz/grid with coordinates/major step}}
        \pgfmathsetmacro{\grd@yc}{\grd@ya + \pgfkeysvalueof{/tikz/grid with coordinates/major step}}
        \foreach \x in {\grd@xa,\grd@xc,...,\grd@xb}
        \node[anchor=north] at (\x,\grd@ya) {\pgfmathprintnumber{\x}};
        \foreach \y in {\grd@ya,\grd@yc,...,\grd@yb}
        \node[anchor=east] at (\grd@xa,\y) {\pgfmathprintnumber{\y}};
      }
    }
  },
  minor help lines/.style={
    help lines,
    step=\pgfkeysvalueof{/tikz/grid with coordinates/minor step}
  },
  major help lines/.style={
    help lines,
    line width=\pgfkeysvalueof{/tikz/grid with coordinates/major line width},
    step=\pgfkeysvalueof{/tikz/grid with coordinates/major step}
  },
  grid with coordinates/.cd,
  minor step/.initial=.2,
  major step/.initial=1,
  major line width/.initial=2pt,
}

% Colors
\usepackage{color} % To define colours?

% Commands

% Other
\renewcommand{\eqref}[1]{Eq. \ref{#1}}
\newcommand{\figref}[1]{Figure \ref{#1}}
\newcommand{\bref}[1]{(\ref{#1})}
\newcommand{\tabref}[1]{Table \ref{#1}}
\newcommand{\appref}[1]{Appendix \ref{#1}}
\newcommand{\secref}[1]{Section \ref{#1}}

% Technologies
\newcommand{\CCGT}{CCGT\xspace}
\newcommand{\OCGT}{OCGT\xspace}
\newcommand{\WindOnsh}{Onshore Wind\xspace}
\newcommand{\WindOffsh}{Offshore Wind\xspace}
\renewcommand{\PV}{Solar\xspace}
\newcommand{\Battery}{Short Term\xspace}
\newcommand{\LTS}{Long Term\xspace}

% Storage operations
\newcommand{\mincost}{\emph{Min Cost}\xspace}
\newcommand{\ddvoll}{\emph{DDVOLL}\xspace}
\newcommand{\minlol}{\emph{Min LOL}\xspace}
\newcommand{\maxlol}{\emph{Max LOL}\xspace}
\newcommand{\minpeak}{\emph{Min Peak}\xspace}
\newcommand{\greedy}{\emph{Greedy}\xspace}

% Capacity credits
\newcommand{\avCC}{\emph{AvCC}\xspace}
\newcommand{\efceens}{\emph{EFC-EENS}\xspace}
\newcommand{\efclole}{\emph{EFC-LOLE}\xspace}

% Expectation operator
\newcommand{\Expct}{\mathop{\mathbb{E}}}

% Commands for comments
\usepackage{ulem} % For striking through text
\definecolor{SG}{rgb}{0,0,0} % Once it's done
\definecolor{citecol}{rgb}{0.5,0.5,0.5} % Once it's done
% \definecolor{SG}{rgb}{1,0,0} % Change later
\newcommand{\edst}[1]{} % For getting rid of strikethroughs
\newcommand{\ed}[1]{\textcolor{SG}{#1}}
% \textcolor{SG}{#1}
% \newcommand{\edst}[1]{\ed{\sout{#1}}}

\newcommand{\addcite}[1]{[\textcolor{citecol}{#1}]}

\newcommand{\paramessage}[1]{#1}
% \newcommand{\paramessage}[1]{1}

% \newcommand{\hide}[1]{\itshape #1 \normalfont}
\newcommand{\hide}[1]{}
